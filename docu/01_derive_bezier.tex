\section{Derivation of a point on a bezier Curve}

\subsection{Calculating the point}
Suppose we have these points as input for th bezier curve.

\begin{equation*}
    \text{BezierPoints}=\left\{B_0,B_1,B_2,B_3\right\}
\end{equation*}
\\
We do the first interpolation between $B_0$, ... $B_3$ with the parameter $\alpha$.

\begin{equation*}
    \left(
    \begin{array}{c}
        Q_0 \\
        Q_1 \\
        Q_2 \\
    \end{array}
    \right)=\left(
    \begin{array}{c}
        (1-\alpha ) B_0+\alpha  B_1 \\
        (1-\alpha ) B_1+\alpha  B_2 \\
        (1-\alpha ) B_2+\alpha  B_3 \\
    \end{array}
    \right)
\end{equation*}
\\
We do the second interpolation between $Q_0$, ... $Q_2$ with the parameter $\alpha$.

\begin{equation*}
    \left(
    \begin{array}{c}
        R_0 \\
        R_1 \\
    \end{array}
    \right)=\left(
    \begin{array}{c}
        (1-\alpha ) Q_0+\alpha  Q_1 \\
        (1-\alpha ) Q_1+\alpha  Q_2 \\
    \end{array}
    \right)
\end{equation*}
\\
We do the final interpolation between $R_0$, and $R_1$ with the parameter $\alpha$.

\begin{equation*}
    S=(1-\alpha ) R_0+\alpha  R_1
\end{equation*}
\\
When we expand these term with respect to $B_0$ ... $B_3$, we get.

\begin{equation*}
    P=\alpha ^3 \left(-B_0\right)+3 \alpha ^3 B_1-3 \alpha ^3 B_2+\alpha ^3 B_3+ \\
    3 \alpha ^2 B_0-6 \alpha ^2 B_1+3 \alpha ^2 B_2                              \\
    -3 \alpha  B_0+3 \alpha  B_1+                                                \\
    B_0
\end{equation*}
\\
factor out the $\alpha$ terms

\begin{equation*}
    P=\alpha ^3 \left(-B_0+3 B_1-3 B_2+B_3\right) \\
    +\alpha ^2 \left(3 B_0-6 B_1+3 B_2\right)     \\
    +\alpha ^1 \left(3 B_1-3 B_0\right)           \\
    +\alpha ^0 B_0
\end{equation*}
\\
and insert missing Points with a multiple of 0.

\begin{equation*}
    P=\alpha ^3 \left(-1 B_0+3 B_1-3 B_2+1 B_3\right) \\
    +\alpha ^2 \left( 3 B_0-6 B_1+3 B_2+0 B_3\right)  \\
    +\alpha ^1 \left(-3 B_0+3 B_1+0 B_2+0 B_3\right)  \\
    +\alpha ^0 \left( 1 B_0+0 B_1+0 B_2+0 B_3\right)
\end{equation*}
\\
From this we can decompose the equation into 3 parts.

\begin{equation*}
    P=\left(
    \begin{array}{cccc}
        B_0 & B_1 & B_2 & B_3 \\
    \end{array}
    \right)
    \times \\
    \left(
    \begin{array}{cccc}
        -1 & 3  & -3 & 1 \\
        3  & -6 & 3  & 0 \\
        -3 & 3  & 0  & 0 \\
        1  & 0  & 0  & 0 \\
    \end{array}
    \right)
    \times \\
    \left(
    \begin{array}{cccc}
        \alpha ^3 \\
        \alpha ^2 \\
        \alpha ^1 \\
        \alpha ^0 \\
    \end{array}
    \right)
\end{equation*}
\\
The left term is our input,
\\
the middle the condesation of our bezier points and
\\
the right term is the parametrization for the curve.
\\\\
Note, that the right term is also the only part of the equation that is responsible for derivatives!
\\\\
With this is mind we can easily calculate derivatives by diffrentiate against $\alpha$.

\begin{equation}
    Derivative_0=\left(
    \begin{array}{cccc}
        \alpha ^3 \\
        \alpha ^2 \\
        \alpha ^1 \\
        \alpha ^0 \\
    \end{array}
    \right)
\end{equation}

\begin{equation}
    Derivative_1=\left(
    \begin{array}{cccc}
        3 \alpha ^2 \\
        2 \alpha ^1 \\
        \alpha ^0   \\
        0           \\
    \end{array}
    \right)
\end{equation}

\begin{equation}
    Derivative_2=\left(
    \begin{array}{cccc}
        6 \alpha ^1 \\
        2 \alpha ^0 \\
        0           \\
        0           \\
    \end{array}
    \right)
\end{equation}

\begin{equation}
    Derivative_3=\left(
    \begin{array}{cccc}
        6 \\
        0 \\
        0 \\
        0 \\
    \end{array}
    \right)
\end{equation}

\subsection{Weighted Points}

The weighted points represent the Bézier points in relation to the time of evaluation and according to the degree of the Bézier curve.
\\\\
The Weighted points can be precalculated to have  less work when evaluation th bezier at some $\alpha$.

\begin{equation*}
    \left(
    \begin{array}{cccc}
        W_0 & W_1 & W_2 & W_3 \\
    \end{array}
    \right)
    =
    \left(
    \begin{array}{cccc}
        B_0 & B_1 & B_2 & B_3 \\
    \end{array}
    \right)
    \times
    \left(
    \begin{array}{cccc}
        -1 & 3  & -3 & 1 \\
        3  & -6 & 3  & 0 \\
        -3 & 3  & 0  & 0 \\
        1  & 0  & 0  & 0 \\
    \end{array}
    \right)
\end{equation*}
\\
or in general
\\
\begin{equation*}
    \left(
    \begin{array}{cccc}
        W_0 & W_1 & W_2 & W_3 \\
    \end{array}
    \right)
    =
    \left(
    \begin{array}{cccc}
        B_0 & B_1 & B_2 & B_3 \\
    \end{array}
    \right)
    \times
    W
\end{equation*}
\\
where $W$ is the weight matrix.
\\\\
With this the calculation of the bezier curve is reduced to a matrix multiplication.

\begin{equation*}
    P=\left(
    \begin{array}{cccc}
        W_0 & W_1 & W_2 & W_3 \\
    \end{array}
    \right)
    \times
    \left(
    \begin{array}{cccc}
        \alpha ^3 \\
        \alpha ^2 \\
        \alpha ^1 \\
        \alpha ^0 \\
    \end{array}
    \right)
\end{equation*}

\subsection{Generalize weight matrix}

We saw the matrix

\begin{equation*}
    \left(
    \begin{array}{cccc}
        -1 & 3  & -3 & 1 \\
        3  & -6 & 3  & 0 \\
        -3 & 3  & 0  & 0 \\
        1  & 0  & 0  & 0 \\
    \end{array}
    \right)
\end{equation*}
\\
This matrix condesens the bezier points into a set of points, that have been weighted according to the Degree of the bezier.
\\
\\
This matrix can be generalized without further proof to

\begin{equation*}
    W=
    \left(
    \begin{array}{cccc}
        W_{0,0}   & W_{1,0}   & \cdots & W_{G-1,0}   \\
        W_{0,1}   & W_{1,1}   & \cdots & W_{G-1,1}   \\
        \vdots    & \vdots    & \ddots & \vdots      \\
        W_{0,G-1} & W_{1,G-1} & \cdots & W_{G-1,G-1} \\
    \end{array}
    \right)
\end{equation*}
\\
where
\\
\begin{equation*}
    W_{i,j}=\begin{cases}
        (-1)^{G+i+j+1} {G-j \choose i} {G \choose j} & i + j \le G      \\
        0                                            & \text{otherwise}
    \end{cases}
\end{equation*}

\subsection{Decomposing Derivatives}

\begin{equation*}
    \left(
    \begin{array}{cccc}
        \alpha ^3 \\
        \alpha ^2 \\
        \alpha ^1 \\
        \alpha ^0 \\
    \end{array}
    \right)
\end{equation*}
\\
can be decomposed into this form

\begin{equation*}
    Derivative_0=
    \left(
    \begin{array}{cccc}
        1 & 1 & 1 & 1 \\
    \end{array}
    \right)
    \times
    \left(
    \begin{array}{cccc}
        \alpha ^3 \\
        \alpha ^2 \\
        \alpha ^1 \\
        \alpha ^0 \\
    \end{array}
    \right)
\end{equation*}

\begin{equation*}
    Derivative_1=
    \left(
    \begin{array}{cccc}
        3 & 2 & 1 & 0 \\
    \end{array}
    \right)
    \times
    \left(
    \begin{array}{cccc}
        \alpha ^2 \\
        \alpha ^1 \\
        \alpha ^0 \\
        1         \\
    \end{array}
    \right)
\end{equation*}

\begin{equation*}
    ...
\end{equation*}

\subsection{Generalize Derivatives}

In general we can calculate the derivatives with the following formula

\begin{equation*}
    D=C \times V
\end{equation*}
\\
where $C$ is the coefficient vector and $V$ is the variable vector.\\
\\
Next we derive the general form, where

\begin{itemize}
    \item $G$ is the degree of the bezier curve
    \item $d$ is th derivative and
    \item $i$ is the i-th factor.
\end{itemize}

\subsubsection{Coefficient vector C}

\begin{equation*}
    C_{d}=
    \left(
    \begin{array}{cccc}
            C_{d,0} & C_{d,1} & ... & C_{d,G - 1} \\
        \end{array}
    \right)
\end{equation*}
\\
where

\begin{equation*}
    C_{d,i}=
    \begin{cases}
        0                                            & d + i \leq G     \\
        1                                            & d>0              \\
        \mathlarger{\prod}_{k = i}^{d + i - 1} G - k & \text{otherwise}
    \end{cases}
\end{equation*}

\subsubsection{Variable vector V}

\begin{equation*}
    V_{d}=
    \left(
    \begin{array}{cccc}
            V_{d,0}   \\
            V_{d,0}   \\
            ...       \\
            V_{d,G-1} \\
        \end{array}
    \right)
\end{equation*}
\\
where

\begin{equation*}
    V_{d,i}=\begin{cases}
        1                  & d + i \leq G     \\
        \alpha^{G - d - i} & \text{otherwise}
    \end{cases}
\end{equation*}

\subsection{Putting it together}

\begin{equation*}
    P=
    \left(
    \begin{array}{cccc}
        B_0 & B_1 & B_2 & B_3 \\
    \end{array}
    \right)
    \times
    W
    \times
    C
    \times
    V
\end{equation*}
\\
where
\\
$W$ is the weight matrix,\\
$C$ is the coefficient vector and\\
$V$ is the variable vector.\\
\\
$C$ and $V$ are responsible for the derivatives of the bezier curve.

\subsection{Compile constant matrices}

$W$ and\\
$C$ \\
are compile constant, because we know the degree and the derivative at compile time.
\\\\
$V$ on the other hand must be generated with respect to $\alpha$, which may not be knowen at compile time.

\subsection{Dimension of the bezier points}

Since know we only have looked at one dimensional points.\\
We can easily generalize this to n-dimensional points, by pluging n-dimensional points into the equation.

\begin{equation*}
    \left(
    \begin{array}{cccc}
        P_x    \\
        P_y    \\
        P_z    \\
        \vdots \\
    \end{array}
    \right)
    =
    \left(
    \begin{array}{cccc}
        B_{x0} & B_{x1} & B_{x2} & B_{x3} \\
        B_{y0} & B_{y1} & B_{y2} & B_{y3} \\
        B_{z0} & B_{z1} & B_{z2} & B_{z3} \\
        \vdots & \vdots & \vdots & \vdots \\
    \end{array}
    \right)
    \times
    W
    \times
    C
    \times
    V
\end{equation*}