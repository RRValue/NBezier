\section{Proof general form of bezier curve}
\label{sec:proofgeneralformofbeziercurve}

\subsection{Definition}
Let $P_{B_0,B_1,\ldots,B_n}$ denote a bezier curve by any selection of points $B_0,B_1,\ldots,B_n$.
Then to start,

$P_{B_0}(x)=B_0$, and

$P(x)=P_{B_0,B_1,\ldots,B_n}(x)=(1-x)P_{B_0,B_1,\ldots,B_{n-1}}+xP_{B_1,B_1,\ldots,B_n}$

\subsection{Proof}
We proof the general form of a bezier curve by induction.
\\
We define, that $P_k$ is a bezier curve of degree $k$.
\\
\\
\textbf{Initial case}
\\
\begin{equation*}
    P_0(x)=B_0
\end{equation*}
\\
\textbf{Hypothesis}
\\
\begin{equation*}
    P_n(x)=\mathlarger{\sum}_{i = 0}^{n} {n \choose i}x^i (1-x)^{n-i} B_i
\end{equation*}
\\
\textbf{Claim}
\\
\begin{equation*}
    P_{n+1}(x)=\mathlarger{\sum}_{i = 0}^{n+1} {n+1 \choose i}x^i (1-x)^{n+1-i} B_i
\end{equation*}
\\
\textbf{Induction step}
\\
From the defition we can write

\begin{equation*}
    P_{n+1}(x)=(1-x)\mathlarger{\sum}_{i = 0}^{n} {n \choose i}x^i (1-x)^{n-i}B_i + x\mathlarger{\sum}_{i = 0}^{n} {n \choose i}x^i (1-x)^{n-i}B_{i+1}
\end{equation*}
\\
We expand this equation

\begin{equation*}
    \mathlarger{\sum}_{i = 0}^{n} {n \choose i}x^i (1-x)^{n-i}B_i -\mathlarger{\sum}_{i = 0}^{n} {n \choose i}x^{i+1} (1-x)^{n-i}B_i + \mathlarger{\sum}_{i = 0}^{n} {n \choose i}x^{i+1} (1-x)^{n-i} B_{i+1}
\end{equation*}
\\
We expand further an get
\begin{align*}
    = & {n \choose 0}x^{0} (1-x)^{n}B_0 & + & {n \choose 1}x^1 (1-x)^{n-1}B_1 & + & \ldots & + & {n \choose n}  x^{n}   (1-x)^{0}B_n &   & \\
    - & {n \choose 0}x^{1} (1-x)^{n}B_0 & - & {n \choose 1}x^2 (1-x)^{n-1}B_1 & - & \ldots & - & {n \choose n}  x^{n+1} (1-x)^{0}B_n &   & \\
    + &                                 & + & {n \choose 0}x^1 (1-x)^{n}  B_1 & + & \ldots & + & {n \choose n-1}x^{n}   (1-x)^{1}B_n & + & {n \choose n}x^{n+1} (1-x)^{0} B_{n+1} \\
\end{align*}
\\
Now we can readout the single $B_k$ terms and reformulate them.
\\
For the $B_0$'s we can write

\begin{equation*}
    {n \choose 0}x^0 (1-x)^n B_0 - {n \choose 0}x^1 (1-x)^n B_0
\end{equation*}
\begin{equation*}
    =(1-x){n \choose 0}x^0 (1-x)^n B_0
\end{equation*}
\begin{equation*}
    ={n \choose 0}x^0 (1-x)^{n+1}B_0 =\end{equation*}
\begin{equation*}
    ={n+1 \choose 0}x^0 (1-x)^{n+1}B_0
\end{equation*}
\\
For the $B_k$'s where $k\in {1, \ldots, n}$ we can write.

\begin{equation*}
    {n \choose k}x^k (1-x)^{n-k} B_k - {n \choose k}x^{k+1} (1-x)^{n-k} B_k + {n \choose k - 1}x^k (1-x)^{n-k+1} B_k
\end{equation*}
\begin{equation*}
    =({n \choose k}x^k (1-x)^{n-k} - {n \choose k}x^{k+1} (1-x)^{n-k} + {n \choose k - 1}x^k (1-x)^{n-k+1})B_k
\end{equation*}
\begin{equation*}
    =((1-x){n \choose k}x^k (1-x)^{n-k} + {n \choose k - 1}x^k (1-x)^{n-k+1})B_k
\end{equation*}
\begin{equation*}
    =({n \choose k}x^k (1-x)^{n-k+1} + {n \choose k - 1}x^k (1-x)^{n-k+1})B_k
\end{equation*}
\begin{equation*}
    =({n \choose k} + {n \choose k - 1})x^k(1-x)^{n-k+1}B_k
\end{equation*}
\begin{equation*}
    ={n+1 \choose k}x^k(1-x)^{n-k+1}B_k
\end{equation*}
\\
For the $B_{n+1}$'s we can write.

\begin{equation*}
    {n \choose n}x^{n+1} (1-x)^0 B_{n+1}
\end{equation*}
\begin{equation*}
    ={n+1 \choose n+1}x^{n+1} (1-x)^0 B_{n+1}
\end{equation*}
\\
we can now sum up every $B_k$, wher $k \in {0, \ldots, n+1}$
\begin{gather*}
    {n+1 \choose 0}x^0 (1-x)^{n+1}B_0 + {n+1 \choose 1}x^1(1-x)^{n}B_1 + \ldots + {n+1 \choose n}x^n(1-x)^{1}B_n + {n+1 \choose n+1}x^{n+1} (1-x)^0 B_{n+1}
\end{gather*}
\begin{equation*}
    = P_{n+1}(x)=\mathlarger{\sum}_{i = 0}^{n+1} {n+1 \choose i}x^i (1-x)^{n+1-i} B_i
\end{equation*}
\\
which is equal to our claim. $ \blacksquare$